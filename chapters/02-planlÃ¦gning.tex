\documentclass[../main]{subfiles}
\begin{document}

\chapter{Systemudvikling}


\section{Virksomhedsbeskrivelse}
\subsection{Interessent analyse}


\section{Projektplan}
    \subsection{Risikoanalyse}


\section{Kravspecifikation}
    \subsection{Kravindsamling}
        Kravene er blevet fundet gennem samtaler med medlemmer og medarbejdere i Fountain House, samt mine erfaringer som medlem i Bogstøtten. Her er de kort opridset som funktionelle og ikke-funktionelle krav.


    \subsection{Funktionelle krav}
        Med systemet skal en bruger kunne\dots
        \begin{itemize}
            \item oprette en konto.
            \item knytte et navn og billede til deres konto.
            \item oprette en begivenhed.
            \item knytte en beskrivelse til en begivenhed.
            \item tilføje opdateringer til en begivenhed.
            \item tilmelde sig en begivenhed.
            \item vise brugere som har tilmeldt sig en begivenhed.
            \item se de nyeste begivenheder.
            \item se begivenheder som de har tilmeldt sig.
            \item se opdateringer på de begivenheder som de har tilmeldt sig.
        \end{itemize}


    \subsection{Ikke-funktionelle krav}
        Systemet skal bedst muligt\dots
        \begin{itemize}
            \item være robust over for brugerinput.
            \item ligne eksisterende webside.
            \item have et simpelt og hurtigt flow, der er allerede en \enquote{god nok} løsning.
            \item følge god praksis i forhold til sikkerhed
        \end{itemize}


    \subsection{User Stories}
        Her er nogle af de funktionelle krav blevet omskrevet til User Stories, der er blevet udeladt de mest basale krav relateret til konto-håndtering.

        \begin{description}
            \item[Som] medarbejder
            \item[vil jeg] oprette en begivenhed med beskrivelse
            \item[fordi] andre skal vide hvad der foregår i huset
        \end{description}

        \begin{description}
            \item[Som] medlem
            \item[vil jeg] se de nyeste begivenheder
            \item[fordi] så ved jeg hvad der foregår i huset 
        \end{description}

        \begin{description}
            \item[Som] medarbejder
            \item[vil jeg] se tilmeldte medlemmer
            \item[fordi] så ved hvor mange der deltager
        \end{description}

        \begin{description}
            \item[Som] medarbejder
            \item[vil jeg] skrive en opdatering på en begivenhed
            \item[fordi] så jeg kan kommunikere ændringer
        \end{description}

        \begin{description}
            \item[Som] medlem
            \item[vil jeg] se opdateringer fra tilmeldte begivenheder
            \item[fordi] så jeg ved hvis der er ændringer
        \end{description}

        \begin{description}
            \item[Som] medlem
            \item[vil jeg] kommentere på begivenheder
            \item[fordi] så andre kan høre mine input
        \end{description}


    \subsection{ER-diagram}
        I ånd med at skære ned på artefakter springer jeg domænemodellen over og producerer et ER-diagram. Efter kravindsamlingen og analyses har jeg dannet et godt indtryk af de dele som systemet består af.
        \todo[inline]{ret prisma fil og lav et ER-diagram}


\end{document}